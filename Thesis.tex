%%%%%%%%%%%%%%%%%%%%%%%%%%%%%%%%%%%%%%%%%%%%%%%%%%%%%%%%%%%%%%%%%%%%%%%%%%%%%%%%
%         Gautam Buddha University, Greater Noida, Utter Pradesh 
%                      https://gbu.ac.in
%               https://www.gbu.ac.in/soict/index.html
%          SOICT M.Tech  Thesis Template — Custom LaTeX Class
%         Created by: [Your Name] — Share & Improve via GitHub
%    GitHub Repository: https://github.com/ashish-kus/SOICT-Latex-Kit
%    author : Ashish Kushwaha , ashish.kus2408@gmain.com, 7525874187
%%%%%%%%%%%%%%%%%%%%%%%%%%%%%%%%%%%%%%%%%%%%%%%%%%%%%%%%%%%%%%%%%%%%%%%%%%%%%%%%

\documentclass{SOICTthesis}  % Load your custom thesis class (.cls file)

% ========================
% Thesis Information Setup
% ========================
\thesistitle{Your Thesis Ti sadas dasd asdtle Here}
\degree{DEGREE NAME}
\program{PROGRAM NAME}
\specialization{SPECIALIZATION}
\authorname{Your Name}
\enrollmentno{Your Enrollment Number}
\supervisor{Supervisor Name}{Designation, Department}
\university{University Name}
\submissiondate{Month Year}

% ================
% Packages
% ================
\RequirePackage{lipsum}     % For generating dummy placeholder text (optional)

\begin{document}

% ====================
% Abstract Section
% ====================
% Set your abstract inside the \theabstract{} command.
\theabstract{
    \lipsum[1-2]  % Replace with your real abstract
}

% ====================
% Chapters Begin Here
% ====================

% ---------------------------------
\chapter{INTRODUCTION}
% ---------------------------------
\section{Background}
\lipsum[3-5]  % Background information
\cite{b1, b3}

\subsection{Ashish}
\lipsum[3-5]  % Background information

\section{Problem Statement}
\lipsum[6-7]
\cite{b2}

\section{Research Objectives}
% Present your research goals here.
\begin{itemize}
    \item \lipsum[8]
    \item \lipsum[9]
    \item \lipsum[10]
\end{itemize}

\section{Scope of Study}
\lipsum[11-12]

% ---------------------------------
\chapter{LITERATURE REVIEW}
% ---------------------------------
\section{Introduction}
\lipsum[13-15]
\cite{b5, b7}

\section{Existing Approaches}
\lipsum[16-18]
\cite{b6, b8}

\section{Research Gaps}
\lipsum[19-21]

% ---------------------------------
\chapter{METHODOLOGY}
% ---------------------------------
\section{Proposed Approach}
\lipsum[22-24]
\cite{b9}

\section{System Architecture}
% You can include system architecture diagram here.
%\begin{figure}[H]
%   \centering
%   \includegraphics[width=0.8\linewidth]{Assets/system_architecture.png}
%   \caption{System Architecture Diagram}
%   \label{fig:system_architecture}
%\end{figure}

\section{Implementation Details}
\lipsum[25-27]

% ---------------------------------------------
\chapter{EXPERIMENTAL SETUP AND RESULTS}
% ---------------------------------------------
\section{Experimental Testbed}
\lipsum[28-30]

\begin{table}[H]
\centering
\caption{Experimental Setup}
\begin{tabular}{|l|l|}
    \hline
    \textbf{Component} & \textbf{Specification} \\
    \hline
    CPU & Example Processor \\
    RAM & Example Memory \\
    OS  & Example OS \\
    \hline
\end{tabular}
\end{table}

\section{Performance Evaluation}
\lipsum[31-33]
\cite{b10, b11}

% Example chart placeholder
%\begin{figure}[H]
%   \centering
%   \includegraphics[width=0.8\linewidth]{Assets/performance_graph.png}
%   \caption{Performance Evaluation Graph}
%   \label{fig:performance_graph}
%\end{figure}

% ----------------------
\chapter{DISCUSSION}
% ----------------------
\section{Findings Analysis}
\lipsum[34-36]

\section{Comparison with Existing Approaches}
\lipsum[37-39]
\cite{b12}

% ----------------------
\chapter{CONCLUSION}
% ----------------------
\section{Summary of Findings}
\lipsum[40-42]

\section{Limitations and Future Work}
\lipsum[43-45]

% =====================
% Bibliography Section
% =====================
\bibliographystyle{IEEEtran}
\newpage
\bibliography{Bibliography/references}

% =====================
% Optional Appendices
% =====================
% Uncomment to include appendices
%\appendix
%\chapter{Appendix A}
%\include{Appendices/AppendixA}

\end{document}

